\documentclass{report}
\usepackage{apacite}
\begin{document}

\chapter{Time Stone}

\section{Extracts from wiki}

"The Time Stone was one of the six Infinity Stones... ...the user is able to physically control and redirect the flow of time, and can specifically select the exact area to manipulate without affecting those outside its selected range. The stone can alter targets as small as an apple or as wide in scope as the timeline of the universe... ...Due to the selective nature of the Time Stone's power, it could be used to individually alter the timeline of individual objects or events, reversing them to a previous state or sending the object forward into a future state. This occurs regardless of any potential breaches in causality... ...the Stone could send objects forward to a potential future state that does not necessarily have to occur in the current timeline."

\section{Abilities}

\begin{itemize}
  \item Can alter an arbitrary space.
  \item Can reverse such a space to a previous state.
  \item Can send the space forward into a future potential state (that does not necessarily have to occur in the current timeline.)
  \item Does not cause issues in causality.
\end{itemize}

\newpage
\section{Definition and Consequences (1st Attempt)}

A TS is a tuple T = (R, $q_{end}$)

\begin{itemize}
  \item R is a deterministic run of some automata which is assumed to halt.
  \item $q_{end}$ the final step of the run.
\end{itemize}

When activated, the TS alters the automaton, sending it forward in time to $q_{end}$. This process is assumed to take no computational power and is a constant $O(1)$ in time and memory complexity.

We define a new MCU complexity class, which in this case defines the number of times the TS must be activated.

\subsection{P = EXPTIME}

Given an arbitrary TM in P, we activate the TS, sending it forward in time to $q_{end}$. Therefore, its running time has been reduced to $O(1)$.

Given an arbitrary TM in EXPTIME, we activate the TS, sending it forward in time to $q_{end}$. Therefore, its running time has been reduced to $O(1)$.

Clearly, given access to a TS, P = EXPTIME.

\subsection{Halting Problem}

In our formal definition, we state the given input TM is assumed to halt. However, this is a very weak assumption to make.

Let's attempt to bypass this requirement and instead see what happens when we run the TS on a TM which does not necessarily halt.

When activated, the TS alters the TM, sending it forward in time to $q_{end}$. This process is assumed to take no computational power and is a constant $O(1)$ in time and memory complexity.

Clearly if the TM does halt, it is sent forward in time to $q_{end}$ as normal. However, what would occur if the TM was an infinite loop? Answering anything other than "the TM is sent forward in time to $q_{end}$ as normal" inherently contradicts the halting problem, as if we define this new behaviour in any way that differs from the original, as a consequence we have created a device which determines if a given TM will halt.

Therefore, this definition cannot exist.

\newpage
\section {Definition and Consequences (2nd Attempt)}

A TS is a tuple T = ($R$, $c$)

\begin{itemize}
  \item $R$ is a deterministic run of some automata which is assumed to halt.
  \item $c$ is a constant multiplier, and accepts the input language of $\mathbf{N}$.
\end{itemize}

When activated, the TS alters the automaton, speeding up its operation by a factor of $c$. This process is assumed to take no computational power and is a constant $O(1)$ in time and memory complexity.

\subsection{P = NP}

We simulate an arbitrary non-deterministic TM that is assumed to halt on a deterministic TM. \cite{simulate-non-determinisim}

The non-deterministic TM can be represented as a tree of non-deterministic choices, where a TM running a BFS simulate non-deterministic behaviour deterministically by iterating through each node at every level of the tree until the TM halts.

As we traverse the BFS tree, the number of nodes at each level increases by a upperbound factor of $n$, where $n$ is the maximum number of non-deterministic choices at some step in the non-deterministic TMs code.

At each level $h$ of the tree, if we activate the TS to speed up the TM by a factor of $n*h$, clearly, we will run through the entire level less than or equal to the time it would take to run a single step.

Therefore, it will now run in the same time as a true non-deterministic TM.

Therefore, when using a TS, P = NP.

\subsection{Halting Problem}

We get any arbitrary TM and modify it such that at every step, an output is given to the TS to double its own multiplier speed. I.e., the first step will take 1 second to compute, the next step will take 0.5 seconds to compute, 0.25s, 0.125s ...

As we know that $1 + 0.5 + 0.25 + 0.125 ... \le 2$, after 2 seconds of computation, we would have ran an infinite number of steps on this TM and will observe two possibilities:

\begin{enumerate}
  \item The TM will have halted as it has completed some $N$ amount of steps.
  \item The TM will continues running as it is in an infinite loop.
\end{enumerate}

Again, the consequence of this concept is we now have a device that determines if a given TM will halt.

Therefore, this definition cannot exist.

\bibliographystyle{apacite}
\bibliography{references}

\end{document}
