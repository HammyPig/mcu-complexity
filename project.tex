\documentclass{report}

\begin{document}

\chapter{Time Stone}

\section{Extracts from wiki}

"The Time Stone was one of the six Infinity Stones... ...the user is able to physically control and redirect the flow of time, and can specifically select the exact area to manipulate without affecting those outside its selected range. The stone can alter targets as small as an apple or as wide in scope as the timeline of the universe... ...Due to the selective nature of the Time Stone's power, it could be used to individually alter the timeline of individual objects or events, reversing them to a previous state or sending the object forward into a future state. This occurs regardless of any potential breaches in causality... ...the Stone could send objects forward to a potential future state that does not necessarily have to occur in the current timeline."

\section{Abilities}

\begin{itemize}
  \item Can alter an arbitrary space.
  \item Can reverse such a space to a previous state.
  \item Can send the space forward into a future potential state (that does not necessarily have to occur in the current timeline.)
  \item Does not cause issues in causality.
\end{itemize}

\newpage
\section{Formal Definition and Consequences (First Attempt)}

A TS is a tuple T = (R, $q_{end}$)

\begin{itemize}
  \item R is a deterministic run of some automata which is assumed to halt.
  \item $q_{end}$ the final step of the run.
\end{itemize}

When activated, the TS alters the automaton, sending it forward in time to $q_{end}$. This process is assumed to take no computational power and is a constant $O(1)$ in time and memory complexity.

We define a new MCU complexity class, which in this case defines the number of times the TS must be activated.

\subsection{P = NP}

We create a proof by doing the following:

\begin{enumerate}
  \item Prove using a TS on an arbitrary deterministic TM that is assumed to halt will result in a time complexity of O(1).
  \item Simulate a non-deterministic TM on a deterministic TM.
  \item Use steps 1 and 2 to show when using the TS, NP problems have a time complexity of O(1), hence P = NP.
\end{enumerate}

\subsubsection{Step 1: TS and deterministic TM interactions}

\textit{Use the time stone to send a deterministic Turing machine forward in time to its end state such that its time complexity is reduced to $O(1)$.}\\

Get any arbitrary deterministic TM that is assumed to halt.

We get the TS components:

\begin{itemize}
  \item R is the deterministic run of the TM.
  \item There exists a $q_{end}$ step, as the algorithm is assumed to halt.
\end{itemize}

We place the TM into its initial step. We activate the TS, sending the TM forward in time to $q_{end}$, where it either halt-accepts or halt-rejects. This takes $O(1)$ time complexity.

The arbitrary deterministic TM, using the TS with an MCU complexity of O(1), has now halted in $O(1)$ time.

We conclude that all deterministic TMs assumed to halt now have a time complexity of $O(1)$ and MCU complexity of $O(1)$ when using a TS.

\subsubsection{Step 2: Simulating a non-deterministic TM using a deterministic TM}

\textit{Have a TM that uses another TM to dictate which path to take at any non-deterministic choice using a BFS, where eventually all paths of the non-deterministic tree are taken. Therefore, the entire process is now deterministic in nature.}\\

We construct a TM containing three tapes.

\subsubsection{The Computer}

The first tape, \textit{the computer}, is designed to do an arbitrary run of some non-deterministic TM that is assumed to halt.

At each non-deterministic step, when facing an $x$ number of choices, each choice is labelled from 0 to $x-1$ such that the step can be chosen deterministically via an external input.

If the input path does not exist, the TM will halt-reject with an additional output specifying that the input path did not exist.

The tape, otherwise, can halt-reject or halt-accept as normal.

\subsubsection{The Echo}

The second tape, \textit{the echo}, simply retains the initial step of \textit{the computer}. It can copy its contents to \textit{the computer}, such that it can return to its initial step after leaving it. Corollary, \textit{the echo} shares the arbitrary tape language of \textit{the computer}.

\subsubsection{The Decider}

The third tape, \textit{the decider}, is designed to determine which route \textit{the computer} will take given a non-deterministic step.

The input language consists of the set of natural numbers, and runs a BFS to determine what choice will be made when given a non-deterministic choice.

\subsubsection{Usage}

When ran, eventually, the BFS will either:
\begin{itemize}
  \item reach a halt-accept state
  \item iterate through the entire tree and find no halt-accept state, and therefore will halt-reject.
\end{itemize}

The non-deterministic TM has now been simulated using a deterministic multi-tape TM. Multi-tape TMs area equivalent to single-tape TMs, therefore we conclude all non-deterministic algorithms that are assumed to halt can be converted to deterministic algorithms that are assumed to halt.

\subsubsection{Step 3: TS and simulated non-deterministic TM interactions}
As we have concluded that all non-deterministic TMs that are assumed to halt can be simulated using a deterministic TM, and all deterministic TMs when using a TS have a time complexity of $O(1)$, we conclude that all non-deterministic TMs which are assumed to halt, when using a TS, have a time complexity of $O(1)$ and MCU complexity of $O(1)$.

Therefore, as all P and NP problems halt, we conclude that given access to a TS, P = NP.

\end{document}
